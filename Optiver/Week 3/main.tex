\documentclass[12pt]{amsart}

\addtolength{\hoffset}{-2.25cm}
\addtolength{\textwidth}{4.5cm}
\addtolength{\voffset}{-2.5cm}
\addtolength{\textheight}{5cm}
\setlength{\parskip}{0pt}
\setlength{\parindent}{15pt}

\usepackage{amsthm}
\usepackage{amsmath}
\usepackage{amssymb}
\usepackage[colorlinks = true, linkcolor = black, citecolor = black, final]{hyperref}

\usepackage{graphicx}
\usepackage{multicol}
\usepackage{ marvosym }
\usepackage{wasysym}
\usepackage{tikz}
\usetikzlibrary{patterns}

\newcommand{\ds}{\displaystyle}
\DeclareMathOperator{\sech}{sech}


\setlength{\parindent}{0in}

\pagestyle{empty}

\begin{document}

\thispagestyle{empty}

{\scshape Optiver Prove it} \hfill {\scshape \large Terminating Dice Sums} \hfill {\scshape Week 3}

\smallskip

\hrule

\bigskip
Let' start by proving that
\begin{equation}
    \lim_{n \to \infty} \left(1 +\frac{1}{n}\right)^{n-1} = e.
\end{equation}
This is fairly trivial, in fact
\begin{equation}
    \lim_{n \to \infty} \left(1 +\frac{1}{n}\right)^{n-1} = \lim_{n \to \infty} \left(1 +\frac{1}{n}\right)^{n-1} \left( 1+\frac{1}{n} \right)  =\lim_{n \to \infty}  \left(1 +\frac{1}{n}\right)^{n} = e
\end{equation}
Where we used the fact that in the reals, the limit of a product of limits is equal to the limit of the products.


Let's move now onto the main problem. We are interest in deciding whether it's more likely to win as the person who starts throwing the dice or not.

Let's define \(P(X=k)\) as the probability of winning at the throw number \(k\).
We can now compute it as follows
\begin{equation}
    P(X = k) = P(X>k-1) - P(X>k).
\end{equation}
Where we used the fact that we are in a discrete setting.
Moreover, we have an explicit formulation for the probabilities above. It follows
that
\begin{equation}
    P(X=k) = \frac{1}{6^k }\left( 6 {5 \choose k-1} - {5 \choose k} \right).
\end{equation}
Computing the value for \(k=2\), we get
\begin{equation}
    P(X=2)=\frac{1}{6^2}20 >\frac{1}{2}.
\end{equation}
As the probability is bigger than \(\frac{1}{2}\) we can already conclude that going as second gives the higher chances of winning.

\end{document}